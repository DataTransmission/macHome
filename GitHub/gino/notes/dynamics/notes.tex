\subsection{Wavenumber, angular frequency, complex plane}
We know that for a wave, $1 (\text{cycle}) = 2\pi (\text{rad}) = 1 (\lambda = x (m))$.
\begin{itemize}
    \item wavenumber spatial frequency of a wave: \\
        \begin{equation}
            k  = \frac{2 \pi \text{(rad)}}{\lambda \text{(m)}} = \frac{1 \text{(cycle)}}{\lambda \text{(m)}}
        \end{equation}
        shows how many cycles per unit distance or radians per unit distance a single wavelength ($\lambda$) has.
        Its a measure of spatial frequency (number of rads or cycles "per" meter), 
        so the longer (shorter) wavelength occurs less (more) frequent in a unit distance.
    \item angular frequency of a wave: \\
        \begin{equation}
            \omega  = \frac{2 \pi \text{(rad)}}{T \text{(m)}} = \frac{1 \text{(cycle)}}{T \text{(m)}},
        \end{equation}
        shows how many cycles or radians a wave has per second.
    \item relationship to complex plane: \\
        The unit circle on the complex plane with $|z|=1$ is $z = e^{i(kx-\omega t)}$. 
        The units for $kx$ is "cycle", thus $k$ is a measure of rate of rotation in a circle as a wave propagates.
        \begin{exmp}
            For a wave of $k=0.1$, it needs to travel $2\pi/0.1= 62.8(m)$ (single wavelength) for a full circle/cycle.
            Whereas $k=10$ only requires $6.28(m)$ to travel a full circle.
            Therefore a wave rotates faster (shorter wavelength) in a complex circle as the $k$ increases.
        \end{exmp}
    \item Visualizing a wave interacting with a complex plane: \\
        Draw a three dimensional cartesian coordinate with $x$ on one axis and the other two are the complex plane axes.
        Draw a wave $f(x)=\cos(kx)$ running with a fixed $k$ to an arbitrary $x$.
        The complex number can be seen as a vector-valued function $z=g(x)=(\cos(kx),\sin(kx))$ mapping $x$ to the complex plane coordinate.
        Thus as the wave propagates along $x$, the $z$ will be spinning in a circle on the complex plane.
        The "graph" $(x,g(x))$ is seen as a spring swirling along/around the $x$ axis.
        The numbers of wave in a unit distance $x=1$ can be seen on the graph $f(0 \ge x \le 1)$.
    \item Visualizing the angular velocity vector (in a rotating reference plane): \\
        A vector that is perpendicular to the rotational circle on a physical plane. 
        Lets focus on three orthogonal vectors, $v$ (cross radial velocity vector), $\omega$ (angular velocity vector), $r$ (radial vector).
        The three coordinates are related by $\omega = \frac{d\theta}{dt} = \frac{1}{r}\frac{dx}{dt} = \frac{v}{r}$, .
        Thus we see that with a constant angular velocity, the smaller $r$ gives higher $v$.
        For an arbitrary velocity vector $\vect{v} = \vect{\omega} \times \vect{r}$. 
\end{itemize}

\subsection{Inertial and non-inertial reference frame}
A rotating reference frame (frame that accelerates w.r.t. an inertial frame) is a non-inertial reference frame.
The Navier-Stokes equation of the absolute velocity (inertial frame) is composed of the 
rotational velocity (rotating frame) plus the velocity contributed by the angular velocity
\begin{equation}
    \Big(\frac{du_{\text{inert}}}{dt}\Big)_{\text{inert}} = \Big(\frac{du_{\text{rot}}}{dt}\Big)_{\text{rot}} + 2\Omega \times u_{\text{rot}}.
\end{equation}
\begin{exmp}
    Suppose a ball going straight from north to south along a longitude when viewed from a fixed point in the outer space.
    This ball will seem to curve right, with westward velocity, when viewed on earth. 
    Thus adding back the eastward rotational velocity will cancel the westward velocity and 
    transform the velocity back to the inertial frame, which is a straight southward velocity.
\end{exmp}

\subsection{Field, Eulerian, Lagrangian}
\begin{itemize}
    \item Eulerian framework (reference frame fixed in space): \\
        Different particles with the QoI measured at the same local point with time.
        This gives the "field" of QoI if all spatial points are measured/modeled. 
    \item Lagrangian framework (reference frame moving in space): \\
        A particle with the QoI is measured by following the particle parametrized by the intial position.
        The instantaeous position $\bf{x}$ becomes a QoI instead of a fixed value as in the Eulerian view.
    \item
\end{itemize}

\subsection{Material derivative/Lagrangian derivative}
The acceleration of a particle in the Eulerian field could be derived from the 
first order Taylor expansion of $u$ by following a particle from $(x,t)$ to $(x+\delta x,t+\delta t)$,
\begin{equation}
    \begin{aligned}
        & u(x+\delta x, t+\delta t) \approx u(x,t) + \delta x \cdot\nabla u + \delta t \frac{\partial u}{\partial t} \\
        & \lim_{\delta t \to 0}\frac{u(x+\delta x, t+\delta t) - u(x,t)}{\delta t} = \frac{\partial u}{\partial t} + u
            \cdot \nabla u. \\
    \end{aligned}
\end{equation}
Imagine a flow with cold water upstream and warm downstream. 
The particle temperature is the scalar QoI. 
Therefore, the particle at any instant is locally heated by the sun which changes the first term.
The changes due to motion (spatial derivative), particle flows from cold to warm with velocity $\bm{u}$,
then it recieves the temperature gradient advection from the second term.

\subsection{Volume material derivative}
A 2D squared volume at the initial time
\begin{equation}
    \delta V = \delta x \delta y = (x_2-x_1)(y_2-y_1).
\end{equation}
At the next time step
\begin{equation}
    \begin{aligned}
    \delta V' & = \delta x' \delta y'  \\
              & = (x_2+u_2\delta t - (x_1+u_1\delta t))(y_2+v_2\delta t - (y_1+v_1\delta t)) \\
              & = (\delta x + \delta u \delta t)(\delta y + \delta v \delta t) \\
              & = \delta V + \delta x \delta v \delta t + \delta y \delta u \delta t + \delta t^2 \delta u \delta v.
    \end{aligned}
\end{equation}
After some algebra
\begin{equation}
    \frac{\Delta \delta V}{\Delta t} = \delta x \delta y (\frac{\partial u}{\partial x} + \frac{\partial v}{\partial y}) + \delta t \delta u \delta v.
\end{equation}
The last term goes to zero as $\delta t \rightarrow 0$,
\begin{equation} \label{eq:volume}
    \frac{D \delta V}{D t} = \delta V \nabla \cdot \vec{u}.
\end{equation}

\subsection{Mass continuity via material derivative}
By definition, mass is a constant/conserved following a particle,
\begin{equation}
    \begin{aligned}
        & \frac{D}{Dt} (\rho \delta V) = 0  \\
        & (\frac{D\rho}{Dt} + \rho\nabla \cdot \vect{v})\delta V = 0 \\
        & \int(\frac{D\rho}{Dt} + \rho\nabla\cdot\vect{v})dV = 0 \\
        & \int(\frac{\partial\rho}{\partial t} + \nabla\cdot(\rho\vect{v}))dV = 0 \\
    \end{aligned}
\end{equation}
Applied \eqref{eq:volume} to the second line.
Applying divergence theorem to the last line gives
\begin{equation}
    \int_V(\frac{\partial\rho}{\partial t})dV = -\int_V(\nabla\cdot(\rho\vect{v}))dV = -\int_S \vect{v} \cdot d\vect{S}
\end{equation}
indicates the increase rate of density inside a particle equals the inflow rate from the surface of the particle.

\subsection{Tangential Vector of a functional surface}
The tangential vector at a point $x$ relative to the point at $y$ is $(y-x,f(y)-f(x))$, where the
first order Taylor expansion of $f(y) = f(x) + \nabla f(x)^T(y-x)$, which gives $(y-x,\nabla f(x)^T(y-x))$.  

\subsection{Tensor}
\subsubsection{Properties}
Tensor, $T$, can be seen as a ``{\bf multilinear function}'' that eats in vectors and spits out a
scalar number, or a ``{\bf linear operator}'' that transforms vectors into vectors. 
Multilinear $f : V_1 \times \cdots \times V_n \rightarrow W$, means linear in each vector space
(closed under additivity and scalar multiplication). \\
{\bf Example: Moment of Inertia tensor ${\bf I}$} \\
\emph{
\begin{equation}
   \text{KE} = \frac{1}{2} (\omega_x, \ \omega_y, \ \omega_z) 
   \begin{pmatrix}
      I_{xx} & I_{xy} & I_{xz} \\
      I_{yx} & I_{yy} & I_{yz} \\
      I_{zx} & I_{zy} & I_{zz} \\
   \end{pmatrix} 
   \begin{pmatrix}
      \omega_x \\ \omega_y \\ \omega_z
   \end{pmatrix}
\end{equation}
where
\begin{equation}
   L =   
   \begin{pmatrix}
      I_{xx} & I_{xy} & I_{xz} \\
      I_{yx} & I_{yy} & I_{yz} \\
      I_{zx} & I_{zy} & I_{zz} \\
   \end{pmatrix} 
   \begin{pmatrix}
      \omega_x \\ \omega_y \\ \omega_z.
   \end{pmatrix}
\end{equation}
The first equation, $I$ is seen as a multilinear function that eats in two vectors of $\omega$ and
spits out a number $2$KE, $I(\omega,\omega) = 2\text{KE}$.
The second equation, $I$ is seen as a linear operator that eats in $\omega$ vector and spits out $L$
vector, $I(\omega) = L$. \\
}
\subsubsection{Components}
A component of $T$ is just a value of the function $T$ on the given basis vectors, e.g. $T_{xx}= T(\hat{x},\hat{x})$. 

\subsubsection{Multilinearity}
Closed under addtivity and scalar multiplication on the separate vector spaces. \\
{\bf Example: rank 2 tensor} \\ 
\emph{
$T: V \times W \Rightarrow \mathcal{R}$.
$T$ is a rank $2$ multilinear function on $v$ and $w$ with basis vectors $\hat{x}_1, \hat{x}_2$ if
\begin{equation*}
   \begin{aligned}
      T(v,w) & = T(v_1\hat{x}_1+v_2\hat{x}_2,w_1\hat{x}_1+w_2\hat{x}_2) \\
             & = v_1w_1T(\hat{x}_1,\hat{x}_1)+ v_1w_2T(\hat{x}_1,\hat{x}_2)+ v_2w_1T(\hat{x}_2,\hat{x}_1)+ v_2w_2T(\hat{x}_2,\hat{x}_2) \\
             & = v_iw_jT_{ij}. 
   \end{aligned}
\end{equation*}
}

%It is a geometric object (like the box for stress tensor) that describe linear relations (dot product, cross product, linear maps) between vectors, scalars and other tensors. 
%It can be multi-dimensional array (e.g., 3D array or more).

%{\bf Examples} \\
%Velocity gradient tensor $\frac{du_i}{dx_j}$, where the scalar $u_i$ (or the vector $u$) is operated by the vector $\nabla$.

%Advective tensor $u_j\frac{du_i}{dx_j}= (\bm u \cdot \nabla \bm u)_i$

\subsubsection{Covariant and Contravariant components of a vectors}
In Cartesian coordinates, the Length of vector $A$, $L_A= A_1^2+A_2^2$ and the dot-product of $A$ and $B$, $A \cdot B= A_1B_1 + A_2B_2$.
In non-Cartesian coordinates, the Length of the vector $A$, $L_A$ and $A\cdot B$ doesn't work, which is why we need covariant and contravariant components. \\
Covariant components for $A$ is $(A_1,A_2)$ and Contravariant is $(A^1,A^2)$.
$L_A= A_1A^1+A_2A^2$.
The Covariant and Contravariant is defined so that $L_A$ and $A\cdot B$ is unchanged.

Original definition of convariant and contravariant: \\
Given two basis, $e$ and $f$, for any Euclidean vector $v= A e= B f$, $A=(a_1,\dotsc,a_n)^T, B=(b_1,\dotsc,b_n)^T$. 
Suppose the transformation matrix, $T$, from $e$ to $f$ satisfies $f= Te$.
Therefore, $v= B T e$, thus $B T= A$, which shows that $B= A T^{-1}$. 
In conclusion, the basis vector $e$ that is transformed covaries with $f$, hence $e$ is transformed covariantly to a new basis $f$, and the components $A$ is transformed contravariantly to the new basis $f$, with respect to the transformation $T$  \\
Covariant transformation: $A e=  BTe= Bf$. \\
Contravariant transformation: $A e= AT^{-1} f= Bf$. \\
\subsection{Upscale KE backscatter}

The $2$D Vorticity equation by streamfunction $\psi$

\begin{equation}
   \frac{D \nabla^2\psi}{D t} = F.
\end{equation}

Suppose there is no $F$, then by multiplying $-\psi$ and integrating over a domain A

\begin{equation}
   \begin{aligned}
      & \int_A \left( -\psi \frac{D\nabla^2\psi}{Dt} \right) dA= 0 \\
      & = \int_A \left( -\psi \nabla\frac{D\nabla\psi}{Dt} \right) dA \\
      & = \int_A \left( -\cancel{\nabla \left( \psi \frac{D \nabla\psi}{Dt} \right)} + \nabla \psi \frac{D\nabla\psi}{Dt} \right) dA \\
      & = \int_A \left( \frac{1}{2} \frac{D(\nabla\psi)^2}{Dt} \right) dA \\ 
      & = \frac{D}{Dt} \int_A \frac{1}{2} (\nabla\psi)^2 dA,
   \end{aligned}
\end{equation}

which gives the domain integrated kinetic energy $E = \int_A\frac{1}{2} (\nabla\psi)^2 dA$ by using periodic boundary condition or $v \cdot n= 0$, and $\frac{D E}{Dt}=0$.
Any function of the vorticity is conserved after integrating over the domain A, $\frac{D}{Dt} \int_A g(\xi) dA= 0$. Therefore 

\begin{equation}
   \begin{aligned}
      & \int_A -\frac{D}{Dt} (\psi \nabla^2\psi) dA= 0= \int_A \left( -\psi \frac{D\nabla^2\psi}{Dt} \right) dA \\
      & \frac{D}{Dt} \int_A -(\psi \nabla^2\psi) dA= \frac{D}{Dt} \int_A \frac{1}{2} (\nabla\psi)^2 dA,
   \end{aligned}
\end{equation}

where $E= -\int_A (\psi \nabla^2\psi) dA$.



\subsection{Pressure Gradient Force}
Given surface depth $Z_1(x,y)$, and the second layer surface $Z_2(x,y)$.
Pressure at a constant depth $z$ in layer $1$ equals $p_1= \rho g (Z_1- z)$.
Pressure at a constant depth $z$ in layer $2$ equals $p_2= \rho g Z_1 + \rho g' (Z_2- z)$.
Therefore, the horizontal pressure gradient at layer $1$ is $\nabla p_1= \rho g Z_1$ where the constant depth $z$ is eliminated.
Likewise, layer $2$ is $\nabla p_2= \nabla p_1 + \rho g' Z_2$. \\

{\bf Remarks}: This suggests the pressure gradient in each layer is just a function of the varying surface height.


\subsection{Flux}
{\bf Definition}: rate of flow of a property (with some unit $U$) per unit area ($U m^{-2}s^{-1}$).








%\end{document}
\subsection{Websites}
\href{http://www2.mpia-hd.mpg.de/homes/dullemon/lectures/hydrodynamicsII/}{Great lecture notes of numerical methods of hydrodynamics}


