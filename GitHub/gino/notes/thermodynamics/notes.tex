
\subsection{State Function}
A function that describes the \emph{equilibrium} state of a system, irrespective of how the system arrived in the state.
e.g, $f(p,V,T)=0$. Whereas, mechnical work and heat depends on the \emph{path} between two equilibrium states. \\

\subsection{Ideal gas}
For $n$ moles of any gas
\begin{equation}
    pV = nR^*T,
\end{equation}
with the universal constant $R^*$ [$J K^{-1}$ mol$^{-1}$].
Suppose $n=1$ mole of dry air, then there is $M_d$ (molecular weight for dry air) grams of dry air
\begin{equation}
        p_d V_d  = \frac{M_d [0.001kg]}{M_d} R^* [J K^{-1} (0.001kg)^{-1}] T = 0.001M_d [kg] \Big(\frac{1000 R^*}{M_d} [J K^{-1} kg^{-1}]\Big) T \\
\end{equation}
which becomes
\begin{equation}
    p_d \alpha_d = R_d T.
\end{equation}
This is the general relationship of any gas, hence the water vapor pressure 
\begin{equation}
   e \alpha_v = R_v T.
\end{equation}

\subsection{Thermodynamic equation}
\begin{equation}
    \frac{DI}{Dt} + p \frac{D\alpha}{Dt} = \dot{Q}
\end{equation}
Q. How to relate the pressure (state) in the momentum equation by the thermodynamic equation? \\
A. $dI = c_v dT \Rightarrow I = c_v T$, and assuming ideal gas $p = \rho R T \Rightarrow p = \rho R I / c_v$,
hence,
\begin{equation}
    \frac{DI}{Dt} - \frac{p}{\rho^2} \frac{D\rho}{Dt} = \dot{Q},
\end{equation}
with the continuity equation it becomes
\begin{equation}
    \frac{DI}{Dt} + \frac{p}{\rho} \nabla \cdot v = \dot{Q}.
\end{equation}
Q. Assuming large scale in \emph{hydrostatic balance}? \\
A. $\alpha d p = -gdz$, hence $dQ = c_v dT - \alpha dp = c_v dT + gdz = d(c_v T + gz) = ds$ (dry static energy)
\begin{equation}
    \frac{Ds}{Dt} = \dot{Q}.
\end{equation}
Q. What about general gas (non-ideal)? \\
A. Diagnostic equation for pressure is $p = -\frac{\partial I}{\partial \alpha}$. \\

\subsection{Entropy}
A measure of ``difference'' between adiabats (no heat exchange processes), and a measure of ``work loss'' in transferring heat (irreversible process). \\

In a closed system, when heat is added at a constant temperature (volume expands and pressure decreases),
the amount of disorder increases which increases the potential temperature (the actual temperature is higher when forced back to original pressure).
For a irreversible process, suppose a heat reservoir at $T_2$ transfers heat $Q$ to a cooler reservoir at $T_0$.
By Carnot's engine, we know that the maximum available energy the heat tranferred into work is
\begin{equation}
    W_2 = (1-\frac{T_0}{T_2})Q.
\end{equation}

For a irreversible process (state variables (e.g., p, V, T) cannot go back to original values without additional work), 
where heat is first transferred to a middle reservoir at the state $T_1 < T_2$, 
and then transfers the same amount of heat $Q$ to the cold reservoir at $T_0$.
Notice since $T_1 < T_2$, therefore the entropy is higher to transfer the same amount of $Q$.
Therefore the maximum available work from the middle reservoir is
\begin{equation}
    W_1 = (1-\frac{T_0}{T_1})Q,
\end{equation}
which is less than $W_2$.
Hence, the work loss from irreversible process is 
\begin{equation}
    W_2-W_1 = Q(T_0/T_1 - T_0/T_2) = T_0(S_1 - S_2) = T_0 dS,
\end{equation} 
in a form of heat loss, where the increase in $dS$ becomes a measure of work loss. 
%A sum of entropy gain of $S_1$ from the middle reservoir and 
%entropy loss of $S_2$ from the original reservoir in the irreversible process.
There is a net gain in entropy in the irreversible process.
i.e., {\bf the more entropy gain/more disorderness contributes to more work loss, which becomes a less efficient process.}

\subsection{specific volume}
{\bf Definition}: $v = 1/\rho$, volume occupied by 1kg of mass.

\subsection{latent heat}
{\bf Definition}: $Q_l$, energy released/absorbed by a body during a constant temperature process (phase transition).


\subsection{Clausius-Claperon Relation}
{\bf Definition}: $\frac{\Delta P}{\Delta T} = \frac{L}{T\Delta v}$, P-T coexistence curve slope relation. \\
Under typical atmoshperic conditions, 
\begin{equation}
   \frac{d e_s}{dT} = \frac{L_v(T)e_s}{R_vT^2}, 
\end{equation}
where: $L_v$ is latent heat for evaporation varying with $T$, 
$R_v$ is the water vapor gas constant. \\

\subsection{saturation vapor pressure}
\noindent August-Roche-Magnus formula (approximation from Clausius-Claperon Relation): \\
\begin{equation}
   e_s(T) = 6.1094 \text{exp}(\frac{17.625T}{T+243.04}).
\end{equation}
( water-holding capacity of the atmosphere increases by about $7\%$ for every $1^{\circ}$C rise in temperature )

\subsection{relative humidity}
{\bf Definition}: $RH = \frac{e}{e_s}(T)$.

\subsection{LCL}
{\bf Definition}: $h_{LC}$, height at which an unsaturated parcel reaches $RH=100\%$ dry adiabatically.

\subsection{dew point temperature}
{\bf Definition}: $T_d$, temperature at LCL when a parcel reaches $RH=100\%$.

\subsection{CIN}
{\bf Definition}: Vertically integrated buoyancy of undiluted cumulus updraft
