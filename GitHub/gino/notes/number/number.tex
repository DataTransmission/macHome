\documentclass[12pt]{article}
\usepackage{setspace}
\doublespacing
\usepackage[margin=0.5in]{geometry}
\usepackage{rotating} % rotate figures to landscape view
%\usepackage[top=15pt, bottom=10pt, left=20pt, right=20pt]{geometry}
\usepackage[toc,page]{appendix}
\usepackage{cancel,comment,alltt}
\usepackage{mathtools,amsmath,amsthm,bm,amsfonts}

\usepackage{url,hyperref,breakurl}

\usepackage{float,subfig,color,array,multirow,tikz}

\usepackage{multirow}
%\linenumbers


\begin{document}

\subsection{Congruence relation (an equivalence relation)}
two integers $a$ and $b$ are congruent modulo n: $a \equiv b$ (mod $n$), with $(a-b)/n= z$, where $z \in Z$.
$n$ is the ``modulus" of the congruence (imagine a clock, $12$ is the modulus).

\subsection{Congruence classes (the equivalence class)}
The equivalent class of the integer $a$ is $\overline{a}_n := \{ \dotsc, a-2n, a-n, a, a+n, a+2n, \dotsc \}$.

\subsection{Integers modulo $n$}
{\bf Def}: The set of all congruence classes of the integers for a modulus $n$

\begin{equation}
  Z/nZ = \{ \overline{a}_n | a \in Z \}.
\end{equation}

When $n \neq 0$ 
\begin{equation}
  Z/nZ = \{ \overline{0}_n, \overline{1}_n, \dotsc, \overline{n-1}_n \}.
\end{equation}

When $n= 0$, $Z/nZ$ is isomorphic to $Z$.

\end{document}
