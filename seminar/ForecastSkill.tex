
	The forecast skill of the ensemble schemes is evaluated by three scores,
	Reliability (REL), Ignorance Skill Score (IGNSS) and Ranked Probability Skill Score (RPSS).\
%	REL (a component of Brier Score for dichotomous events) 
%	measures whether the model is well-calibrated.\
%	IGNSS is a measure of dispersion of the ensemble, it assigns
%	penalty to underdispersed ensembles that cause systematic errors.\
%	RPSS evaluates a multi-category event,
%	it assigns penalty to the ensemble members that are further away from the category of occurrence 
%	(i.e., penalizes with a distance measure).\
%	For RPSS and IGNSS, the forecast states are divided into $10$ 
%	categories, which are the deciles of the climatological distribution.\


	
	% Discuss the overall PCE method doing
	Figure {\ref{scoreTau2}} ($\tau_z=2$) and Figure {\ref{scoreTau10}} ($\tau_z=10$) 
	show the time evolution of three scores 
	RPSS (top), IGNSS (middle) and REL (bottom) generated by the schemes:
	(I) Uninformative perturbed parameter ($e_{\text{unif}}$),
	(II) Informative perturbed parameter ($e_{\text{clim}}$, $e_{\text{AR1}}$), 
	(III) Additive stochastic parameterization (\emph{Stoch}), 
	(IIII) Deterministic parameterization (\emph{Det}). 
	The PCE for the perturbed parameter scheme in the two figures is approximated by $64^{\text{th}}$ degree polynomial.\
	As expected, without a probabilistic feature, \emph{Det} performed poorly over all scores.\
	Therefore, we only compare the \emph{Stoch}, and the perturbed parameter scheme using  
	$e_{\text{clim}}$, $e_{\text{AR1}}$ and $e_{\text{unif}}$ in the following.\


	Results in Figure {\ref{scoreTau2}} and Figure {\ref{scoreTau10}}
	show consistent and nearly identical scores by $e_{\text{clim}}$ and $e_{\text{AR1}}$.\
	This supports the hypothesis of applying informative distributions, regardless of the
	highly bifurcated the states in $\tau_z=10$, for the perturbed parameter scheme.

	At $\tau_z=10$, the uninformative $e_{\text{unif}}$ performs poorly in Figure {\ref{scoreTau10}} (a) and (c).\
	The failure is attributed to the wide ensemble spread (caused by strong bifurcations) at $\tau_z=10$.\
	Suppose the wide-spread case ($\tau_z=10$) and 
	the small-spread case ($\tau_z=2$) both forecast low density 
	at the category of occurrence by using $e_{\text{unif}}$.\
	The small-spread case will remain reliable with the entire ensemble
	contained in (or in close proximity to) the category of occurrence.\
	Whereas the wide-spread case using the uninformative input may have its highest density 
	erroneously far from the category of occurrence, resulting in a penalty in RPSS.\
	This explains why it is necessary to apply informative $e_s$, 
	especially when forecasting the wide-spread cases.\ 

%	At $\tau_z=2$, \emph{Stoch} underperformed in Figure {\ref{scoreTau2}} (c), (d) and (e)
%	with contrasting results in $\tau_z=10$ (Figure {\ref{scoreTau10}}),
%	which are attributed to two possible explanations: 
%	1) At $\tau_z=2$ ($\tau_z=10$), the slow (fast) varying, less (more) white noise like, subgrid-scale dynamics
%	is harder (easier) to be modeled by a stochastic process, and
%	2) the Gaussian stochastic term $\epsilon$ in equation \eqref{AR1} may 
%	not (may) represent a non-Gaussian (Gaussian-like) true residual in Figure {\ref{hist_r}} (a) (in Figure {\ref{hist_r}} (b)).\

%	In order to demonstrate the overall cost reduction by the PCE, 
%	we show a low order ($16^{\text{th}}$ degree polynomial) PCE forecast skill in Figure {\ref{scoreTau2_lev6}}
%	by keeping other schemes unchanged as in Figure {\ref{scoreTau2}}.\
%	The overall performance of the two informative $e_{\text{clim}}$ and $e_{\text{AR1}}$ 
%	are comparable to \emph{Stoch} by using significantly less simulations.\
%	This demonstrates that with a relatively smooth state ($\tau_z=2$), PCE could substantially reduce the computational cost.\
%	The weaker IGNSS (d) and REL (f) at longer lead time ($\ge 10$ MTU) in the ocean component
%	is caused by low order PCE converging slower when the jump discontinuities begin to increase.\
%	Despite the weakness, IGNSS and REL (Figure {\ref{scoreTau2_lev6}} (c) and (e)) 
%	by $e_{\text{clim}}$ and $e_{\text{AR1}}$ remain better than \emph{Stoch} 
%	in the atmosphere with shorter lead times ($\le 5$ MTU).

	
