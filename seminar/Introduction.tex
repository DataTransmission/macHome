
	Operational weather and climate prediction centers rely on
	probabilistic ensemble forecasts to account for model and measurement errors.\ 
%	Assuming the perturbed ensemble states to be normally distributed, 
%	the ensemble mean becomes the best estimate of the true state as it minimizes
%	the forecast errors in a least square sense.\ 
	This study will focus on two widely applied ensemble forecast schemes: 
	a perturbed parameter scheme and a stochastic parameterization (with additive noise).\ 
	Both schemes introduce perturbations to the model parameters to represent uncertainties.\
	For the perturbed parameter scheme, a perturbation is randomly sampled 
	prior to each model integration and held fixed through the simulation.\
	Whereas the stochastic parameterization applies time-varying perturbations 
	to the model parameter throughout the model integration.\ 
	In \citet{Arnold13}, the stochastic parameterizations were shown to 
	produce more reliable forecasts than the steady perturbed parameter counterpart.\
	The fluctuation of perturbations is argued to be important 
	as it represents missing subgrid-scale processes \citep{Palmer01}.\ 
	However, our study shows that the perturbed parameter scheme could be improved to 
	be equally reliable when ``informative" distributions are applied
	(``Informative" is defined in section  [\ref{sec:schemesPert}]).\ 
	The proposed scheme is tested on a Lorenz system
	coupled to a spatially-resolved convective-scale system,
	and produces reliable forecasts as an additive stochastic parameterization.\ 
	The perturbed parameter scheme produces a large ensemble effectively
	with PCE while preserving the forecasts' unbiased statistical properties.\


%	Probabilistic forecasts are used to account 
%	for uncertainties in the forecasting system.\
	The uncertainties are generally from two sources:
	model error and initial condition error.\
	Model errors are caused by bulk parameterization of subgrid-scale processes, 
	discrete integration, inexact representation of physics, etc;
	whereas the initial condition errors are caused by 
	imperfect instruments, lack of continuous  
	measurements to accurately serve a discretized model, etc.\
	The two types of errors are not mutually exclusive, 
	may interact nonlinearly during the model simulation, 
	and may lead to large uncertainties in the forecast states.\
	By using perfect initial conditions in this study, we focus on quantifying only the model errors.\
	The ensemble forecast schemes to represent model errors 
	include the perturbed parameter scheme \citep{Bowler08}, 
	stochastic parameterizations \citep{Buizza99}, multi-model \citep{Kirtman14}
	and multi-parameterization \citep{Stensrud00} schemes.\
	An appropriate representation of the errors will ensure good forecasts.\
	

	The main issue for an ensemble generated by a best single model is the unresolved subgrid-scale processes which cause
	highly correlated underdispersed ensembles leading to systematic bias {\citep{Palmer01, Stensrud00}}  
	(i.e, the ensemble spread does not cover the observation {\citep{Buizza05}}).\
	Despite multi-model ensemble approach captures the true state better by resolving structural uncertainty {\citep{Tebaldi07}},
	the underdispersed states in individual models remain an issue.\
	A recent study of 24 models in CMIP {\citep{Pennell11}} concluded that
	the effective number of independent models is between 7.5 and 9.\ 
	This indicates that the models are not as independent as expected.\
	In addition, quantifying the errors for multiple models remains ad hoc when
	different sources of model errors interplay {\citep{Palmer09}}.
	Multi-parameterization {\citep{Houtekamer96}} using the 
	same dynamical core also suffers from these issues.\ %{\citep{Arnold13}}.\

	Stochastic parameterizations with a best single model, however, show promising results
	in reducing systematic errors and enable quantitative analysis of model errors 
	{\citep{Palmer09, Berner09, Arnold13}}.\
	The two major classes of stochastic parameterizations developed are 
	Stochastic Perturbed Parameterization Tendencies (SPPT) {\citep{Buizza99}}  
	and Stochastic Kinetic Energy Backscatter (SKEB) schemes
	{\citep{Shutts05, Berner09, Bowler09}}.\
	SKEB provides a missing upscale kinetic energy cascade scheme, whereas
	SPPT generates stochastic perturbations in existing bulk-parameterized tendency 
	terms to mimic the subgrid-scale dynamics.
	We applied SPPT as our stochastic parameterization following \citet{Arnold13} 
	and leverage the success to construct a reliable perturbed parameter scheme (also a single model approach). 

	The perturbed parameter scheme applied by \citet{Arnold13},
	using the coupled L96 model, resulted in unreliable forecasts,
	which may be attributed to: 
	1) The use of arbitrary distributions for each perturbed parameter (a total of four parameters).\
	2) Small sample size for each parameter, which does not allow sample statistics 
	to converge probabilistically to the expected values at large sample size (i.e., central limit theorm).\
	The first issue	is analogous to applying a white noise process, instead of an 
	autoregressive process, to the stochastic parameterization,
	which results in unreliable forecasts in \citet{Arnold13}.\
	Therefore, the revised scheme applies an informative perturbed parameter distribution to generate the ensemble.\ 
	The second issue is generally resolved by repetitively integrating the model 
	whenever a perturbation is generated to create a large ensemble, but at a high computational cost.\
	The expense increases significantly as the number of perturbed 
	parameter increases, such as four parameters used by \citet{Arnold13}.\
	Therefore, we propose to build a ``surrogate" model
	through stochastic spectral method, Polynomial Chaos Expansion (PCE), 
	with the ability of fast convergence to the exact state as a function of the perturbation {\citep{Lucor01}}.\ 
	Once built, we need only to evaluate the analytical PCE function at nearly no cost.
%	PCE is effective at generating large ensemble size for a reliable statistics when it
%	approximates the exact ensemble forecast states with minimum error.\
	The PCE received considerable attention in a wide range of engineering applications, 
	such as computational fluid mechanics {\citep{Hosder06}}, 
	and was recently applied to an ocean GCM {\citep{Alex12}}.\

	
	The two innovations of the proposed perturbed parameter scheme introduced here can be summarized as follows
	1) informative perturbed parameter distributions with an 2) efficient sampling surrogate, PCE,
	to generate reliable ensemble forecasts.\



	The experimental setup in Section \ref{sec:expsetup} includes a ``truth model", 
	providing ``true states" for the forecast schemes to utilize and verify.\
	A deterministic parameterization scheme in Section \ref{sec:schemesDet}
	and the additive stochastic parameterization in Section \ref{sec:schemesStoch}
	are formulated both to enable comparison with 
	the PCE-accelerated informative perturbed parameter scheme in Section \ref{sec:schemesPert}.\
	Forecast consistency of the two ensemble forecast schemes is discussed in Section \ref{sec:statConsistency}.\  
	Finally, to conclude the proposed scheme being effective, all forecast schemes are
	compared using a variety of forecast skill scores in Section \ref{sec:skills}.\ 

