
	We have implemented a coupled ocean-atmosphere system (L63) with a subgrid-scale 
	system (L96) in the atmospheric component.\
	The major result is to generate reliable forecasts
	by the proposed perturbed parameter scheme through
	the employment of informative input distributions.
	The informative perturbed parameter scheme especially outperformed an additive stochastic 
	parameterization in a more realistic subgrid-scale system ($\tau_z=2$).
	The low cost PCE surrogate guarantees unbiased statistics at large ensemble size,	
	and effectively delivers the ensemble distributions without the need to integrate the actual forecast model.\ 
	The numerical integration (i.e., quadrature) to build the PCE treats the 
	model as a black box, which is an advantage for operational forecasts using complex GCMs.\
	

%	Due to the strong bifurcation at longer lead times in a fast varying system ($\tau_z=10$), 
%	the estimation of exact states requires higher order PCE.\
%	The multiple large jump discontinuities may not be observed in a realistic dynamical system.\
%	Thus the proposition of PCE being efficient is not violated 
%	when given a relatively smooth and realistic experiment ($\tau_z=2$),
	%or forecasts created at a shorter lead time 
	%($<10$ MTU, or approximately less than $14$ atmospheric days).\
	
	The forecast skill of using just a single perturbed parameter
	for the PCE-accelerated informative perturbed parameter scheme
	is surprisingly competitive with the additive stochastic parameterization.\ 
	The challenge is to apply this to a realistic model with more parameters,
	where our forecast framework is developed to test multiple perturbed parameters effectively.\


