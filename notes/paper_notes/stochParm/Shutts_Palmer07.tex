\documentclass[margin=0.5in,twocolumn,12pt]{article}
\usepackage{rotating} % rotate figures to landscape view
%\usepackage[top=15pt, bottom=10pt, left=20pt, right=20pt]{geometry}
\usepackage[toc,page]{appendix}
\usepackage{cancel,comment,alltt}
\usepackage{mathtools,amsmath,amsthm,bm,amsfonts}
\usepackage{url,hyperref,breakurl}

\usepackage{float,subfig,color,array,multirow,tikz}

\usepackage{multirow}
%\linenumbers
\graphicspath{./}

\newcommand{\onehalf}{\frac{1}{2}}
\newcommand{\nnd}{^{\text{nd}}}
\newcommand{\nth}{^{\text{th}}}


\begin{document}
{\bf{Convective Forcing Flutuations in a CRM: Relevance to the SP Problem}} \\
Coarse-grained variables: \\
$\bar{Q}$, $\bar{Q}_1 = Q_1(\bar{X})$, $\tilde{Q} = \Big(\bar{V}\cdot\nabla\bar{\theta} - \overline{V\cdot\nabla\theta}\Big) + \bar{Q}$ (effective heating)

Coarse model variables: \\
$Q = Q_R + Q_D + Q_{ls} + Q_1$ \\

\begin{enumerate}
\item Since the coarse model has a parameterization for $Q_1(X)$, this function could be used for the coarse-grained outputs to see the effects on $Q_1(\bar{X})$.
Examine the relationship between effective heating  $\tilde{Q}(\bar{X})$ (including all heating, e.g., latent, solar, sensible, convective, etc) and parameterized convective heating $Q_1(\bar{X})$, to see how much the parameterization of convective heating affects the effective heating.

\item $\tilde{Q}$ is usually lower than $Q_1$ alone, since there exists the cooling effect of $Q_{ls}$ in $\tilde{Q}$. 
\end{enumerate}
























\end{document}

