\documentclass[12pt]{article}
\usepackage{setspace}
\doublespacing
\usepackage[margin=0.5in]{geometry}
\usepackage{rotating} % rotate figures to landscape view
%\usepackage[top=15pt, bottom=10pt, left=20pt, right=20pt]{geometry}
\usepackage[toc,page]{appendix}
\usepackage{cancel,comment,alltt}
\usepackage{mathtools,amsmath,amsthm,bm,amsfonts}
\usepackage{url,hyperref,breakurl}

\usepackage{float,subfig,color,array,multirow,tikz}

\usepackage{multirow}
%\linenumbers
\graphicspath{./}

\newcommand{\onehalf}{\frac{1}{2}}
\newcommand{\nnd}{^{\text{nd}}}
\newcommand{\nth}{^{\text{th}}}
\newcommand{\Exp}{{\text{\bf E}}}
\newcommand{\Var}{{\text{\bf Var}}}
\newcommand{\rv}{\color{red}}

\begin{document}
\begin{itemize}   

\item 1. $P_2(\mathcal{R})$ is a inner-product space with $<p,q>= \int_0^1 p(x)q(x)dx$. Define $T \in L(P_2(\mathcal{R}))$ by $T(a_0+a_1x+a_2x^2)=a1x$ on the basis $(1,x,x^2)$.
a) Show that $T$ isn't self-adjoint. \\
Proof: \\
a) The inner product 
$<T(a_0+a_1x+a_2x^2),b_0+b_1x+b_2x^2> = <a_1x,b_0+b_1x+b_2x^2>= \int_0^1a_1(b_0+b_1x+b_2x^2)dx= a_1b_0+1/2a_1b_1+1/3a_1b_2$.
If $T^*=T$, then $<Tv,w>= <v,Tw>$, but
$<a_0+a_1x+a_2x^2,T(b_0+b_1x+b_2x^2)> = <a_0+a_1x+a_2x^2,b_1x>= \int_0^1b_1(a_0+a_1x+a_2x^2)dx= b_1a_0+1/2b_1a_1+1/3b_1a_2$ implies $<Tv,w> \neq <v,Tw>$, hence $T\neq T^*$. \\

b) The reason to the matrix of $T$ wrt the basis $(1,x,x^2)$ equals its conjugate transpose, but still not self-adjoint is due to the basis not being orthonormal over the inner product $\int_0^1 dx$.  ($M(T,(e_1,\dotsc,e_n)) = M(T^*,(e_1,\dotsc,e_n))^T$ if the basis vectors are orthonormal) \\

%The regular inner product is done on the each coordinate separately, but the product over the integrand is done over mixed coordinates. \\

\item 2. Product of two self-adjoint operators are not self-adjoint. \\
Proof: \\
\begin{equation}
\begin{pmatrix}
   1 & 1 \\
   1 & -2
\end{pmatrix}
\begin{pmatrix}
   2 & 1 \\
   1 & 1 
\end{pmatrix}
=
\begin{pmatrix}
   3 & 2 \\
   0 & -1
\end{pmatrix}
\end{equation}

\item 3. The set of self-adjoint operators forms a subspace of $L(V)$. \\
subspace












\end{itemize}
\end{document}
