
\subsection{\bf Prove that a $n$ point Gauss-quadrature is exact for a polynomial up to order $(2n-1)$} 
{\bf Quadrature Exactness for polynomial of order $n-1$}: \\
The integration of the form
\begin{equation}
  I(f) = \int_{a}^{b} \omega(x) f(x) dx
\end{equation}
is exact when 
\begin{equation} \label{eq:qep}
  I(f) \approx Q(f) = \sum_{i=1}^{n} w(x_i) f(x_i).
\end{equation}
A function approximated by the $n$ point Lagrange interpolant can be represented as a $(n-1)$th order polynomial
\begin{equation}
  f(x) \approx \sum_{i=1}^{n} f(x_i) \prod\limits_{j=1,j\neq i}^{n}\frac{x-x_j}{x_i-x_j}
\end{equation}
Substituting into 
\begin{equation}
  \begin{aligned}
    I(f) & = \int \omega(x) f(x) dx  \\
         & \approx \int \omega(x)\sum_{i=1}^{n} f(x_i) \prod\limits_{j=1,j \neq
	 i}^{n}\frac{x-x_j}{x_i-x_j}dx \\
         & = \sum_{i=1}^{n} \Big( \int \omega(x) \prod\limits_{j=1,j \neq
	 i}^{n}\frac{x-x_j}{x_i-x_j} dx \Big) f(x_i) \\
         & = \sum_{i=1}^{n} w(x_i) f(x_i),
  \end{aligned}
\end{equation}
and don't forget $w$ is not a continuous function of the domain $x$, instead only has discrete
values as a function of $x_i$.
Therefore this shows that the {\bf\emph{quadrature is exact when the function $f$ is a polynomial of order
$(n-1)$}}. \\

{\noindent \bf Orthogonality}: \\
Let $p_i$'s be orthogonal polynomials of maximum order $n$ over $[a b]$ such that
\begin{equation}
  \int_{a}^{b} \omega(x) p_i(x) p_j(x) dx = c \delta_{ij},
\end{equation}
where $c$ is a constant factor. 
Any polynomial $h_k(x)$ with order $k \le (n-1)$ is a linear combination
of $p_k$'s such that
\begin{equation} \label{eq:orth}
   \int_a^b \omega(x) h_k(x) p_n(x)dx= 0 \text{ for all } k= 0,1, \dotsc, n-1,
\end{equation}
shows the terms in $h_k$ are all orthogonal to $p_n$. \\

{\noindent \bf Quadrature Exactness for Orthogonality}: \\
The orthogonality integration \eqref{eq:orth} is ``exact'' if it is done over the roots of $p_n(x)$, 
\begin{equation} \label{eq:qe}
  \sum_{i=1}^n \omega(x_i) h_k(x_i)p_n(x_i) = \int_a^b \omega(x) h_k(x) p_n(x)dx = 0.
\end{equation}

{\noindent \bf Proof}: \\
Using the two exactness conditions (\eqref{eq:qep} and \eqref{eq:qe}) and the orthogonal polynomials, we will prove that $I(f) \approx
Q(f)$ for $f$ of order $2n-1$. 
The quotient form of polynomial $f(x) = p_n(x)q(x) + r(x)$, where $q$ and $r$ are both order $\le n-1$,
\begin{equation}
   \begin{aligned}
   I(f) & =  \int_a^b \omega f dx & &  \\
   & = \int_a^b \omega q p_n dx & + \int_a^b \omega r dx \ \ \ \ \ \ \ & \text{ (quotient form) } \\
   & = 0 & + \int_a^b \omega r dx \ \ \ \ \ \ \ & \text{ (orthogonality) }\\
   & \approx \sum w(x_i) q(x_i) p_n(x_i) & + \sum w(x_i) r(x_i) & \text{ (exactness \eqref{eq:qe} + \eqref{eq:qep}) } \\
   & = \sum w(x_i) f(x_i) = Q(f). \\
   \end{aligned}
\end{equation}

{\noindent \bf Concluding Remarks}: \\
\begin{itemize}
  \item The quadrature exactness can go up to order $2n-1$ for $f$, and the weights are on the roots of
    orthogonal polynomials. \\
  \item If $\omega(x)=1$ (associated with the Legendre orthogonal polynomials), then $I(f)=\int f dx$ is the regular integration for $f$.
\end{itemize}
